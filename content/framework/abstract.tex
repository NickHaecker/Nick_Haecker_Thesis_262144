\chapter*{Abstract\markboth{Abstract}{}}
\addcontentsline{toc}{chapter}{Abstract}
\say{\emph{A Fraction of Time (Arbeitstitel)}} basiert auf der Idee, den Spieler verschiedene parallele Zeitlinien von sich selbst erzeugen zu lassen. Der Spieler muss in diesen verschiedenen Zeitlinien, mithilfe der diversen Zeitlinien von sich selbst, Rätsel lösen und mit seinen Zeitlinien zusammenarbeiten. Grundlage dieser Weiterentwicklung ist der digitale Prototyp, der bereits im Gamedesign Workshop im Wintersemester 2021/ 2022 entwickelt wurde. 

Die Zielsetzung dieser Abschlussarbeit ist es, das bisher implementierte System auf technischer Seite so weit zu verbessern, dass es für den Spieler offensichtlich fehlerfrei funktioniert. Außerdem soll ein Rahmen geschaffen werden, in dem es möglich ist, Erweiterungen zu integrieren. Hinzu werden weitere Inhalte, wie weitere Level und Spielwelten konzipiert und umgesetzt. Der Spieler soll dabei eine Variation an Spielwelten und Rätsel erhalten, welche er lösen muss, um das Level abzuschließen. Hierbei soll ihm auch das Spielprinzip so vermittelt werden, dass er durch das Verständnis der Mechanik jedes Rätsel lösen kann. Validiert werden die Inhalte durch User-Tests, bei denen das Verständnis der Spielmechanik auf inhaltlicher und technischer Ebene abgefragt wird.