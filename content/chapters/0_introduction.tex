\chapter{Einleitung}\label{sec:intro}
Das grundlegende Konzept dieser Bachelor-Thesis wurde im Rahmen des Gamedesign-Workshops an der \ac{HFU} im Wintersemester 2021/2022 entworfen.
Durch das gute Feedback am Tag der Medien 2021 und das Interesse an einer echten spielfähigen Version wurde beschlossen, das Konzept weiterzuentwickeln. 
Inhalte aus dem im Anhang beiliegenden Gamedesign-Dokument wurden übernommen, einige wurden verbessert und andere verworfen. 
Der Titel des Projektes heißt \say{\emph{A Fraction of Time}}.


\section{Motivation und Aufgabenstellung}
Das Ziel dieser Bachelorarbeit ist es, den bisherigen digitalen und spielbaren Prototypen aus dem Gamedesign Workshop 2021/2022 auf technischer Ebene neu zu entwickeln. Das System soll zum einen fehlerfrei funktionieren und zum anderen um weitere Rätsel, Dialoge, Interaktionsobjekte und neue Szenarien erweiterbar sein. Zusätzlich soll eine Einführung des Spielers durch ein Tutorial erfolgen. 


Die Leitfragen, an denen sich die Entwicklung und Konzeption des Prototyps orientiert hat, lauten:
\begin{itemize}  
    \item \textbf{Wie kann das Spielkonzept von  \emph{\say{A Fraction of Time}} als leicht verständliches Tutorial dargestellt werden?}
    \item \textbf{Wie komplex können Rätsel aufgebaut werden, sodass der Spieler maximal herausgefordert wird?}
\end{itemize}

\section{Struktur der Arbeit}
Diese Arbeit gibt zunächst in \say{Kapitel \ref{sec:basics}: \nameref{sec:basics}} einen Überblick über die theoretischen Grundlagen des Determinismus, der für die spätere Umsetzung relevant ist. Dabei wird ebenfalls auf Spieltitel eingegangen, die von der Art ähnlich sind. Darauffolgend wird in \say{Kapitel \ref{sec:concept}: \nameref{sec:concept}} auf die grundlegende Konzeption des Spiels eingegangen. Dabei werden Themen angesprochen, die entweder nur für das \ac{UI} Design wichtig sind oder lediglich für die spätere Umsetzung. In \say{Kapitel \ref{sec:design}: \nameref{sec:design}} wird auf die gestalterischen Aspekte des Prototyps eingegangen. In \say{Kapitel \ref{sec:dev}: \nameref{sec:dev}} wird die Umsetzung des Systems erklärt, das anschließend in \say{Kapitel \ref{sec:test}: \nameref{sec:test}} mit geführten User-Tests validiert wird. Abschließend werden ein Ausblick und ein Fazit über diese Arbeit und der dahinter stehenden Entwicklung gegeben.
