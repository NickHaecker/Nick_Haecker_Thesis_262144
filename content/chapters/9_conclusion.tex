\chapter{Fazit}\label{sec:fazit}
In diesem Kapitel werden die Ergebnisse der Arbeit zusammengefasst. Zum Schluss wird ein Ausblick auf die Zukunft des Spiels gegeben.
\section{Zusammenfassung}
In der vorliegenden Arbeit wurde ein verbesserter Prototyp und eine verbesserte Konzeption aus der Grundlage aus dem Gamedesign Workshop 2021/2022 entwickelt. 

% Zunächst wurde dafür in \say{Kapitel \ref{sec:basics}: \nameref{sec:basics}} eine Analyse der Genre ähnlichen Spiele durchgeführt, und die Grundlagen der Kernspielmechanik 
Dafür wurde zunächst in  \say{Kapitel \ref{sec:basics}: \nameref{sec:basics}} eine Analyse von genreähnlichen Spielen sowie Spielen mit einer ähnlichen Spielmechanik durchgeführt, welche in der Spielart Bezüge zu diesem Prototyp aufweisen. Außerdem wurden die Grundlagen der Spielmechanik des Prototyps erklärt.

Im weiteren Schritt wurde das Konzept aus dem Gamedesign Workshop in \say{Kapitel \ref{sec:concept}: \nameref{sec:concept}} überarbeitet und in einen passenden Rahmen gesetzt. Das Konzept umfasst nun die wichtigsten Elemente, die das Spiel auf der konzeptionellen Seite haben sollte. Darunter zählen die Hintergrundgeschichte des Spiels, die Welt in der sich der Spieler bewegt, die Möglichkeiten, die ihm geboten werden, um der Geschichte in der Welt zu folgen und die Hilfsmittel, Informationen und Rätsel, die eine solche Geschichte enthällt.

In \say{Kapitel \ref{sec:design}: \nameref{sec:design}} wurde eine visuelle Gestaltung der Welt erschaffen, die einen ersten Eindruck des Spiels vermitteln soll. Darunter zählen das Aussehen des Chronologen, erste prototypische Mockups des \ac{UI}s sowie Weltgegenstände, mit denen der Spieler interagieren kann nun eine Auswirkung auf weitere Weltgegenstände zu haben. Im Rahmen der Spielwelt wurden die konzipierten und umgesetzten Rätsel vorgestellt, welche der Spieler lösen darf. Hinzukommen die bereits konzipierten aber noch nicht umgesetzten Rätsel. 

In \say{Kapitel \ref{sec:dev}: \nameref{sec:dev}} wurde die Umsetzung der Spielmechanik vorgestellt. Zunächst wurden verwendete Technologien vorgestellt, welche der Entwicklung des Prototyps zugrunde liegen. Im Anschluss wurden zunächst die technischen Mängel des ursprünglichen Prototyps analysiert und festgehalten, welche in der Weiterentwicklung des Prototyps verbessert werden mussten. Das überarbeitete System wurde ebenfalls in seinen Kernelementen vorgestellt und in der Art so angefertigt, dass sie für den Ausblick des Spiels generisch erweiterbar sind.
Außerdem konnte das System hinreichend deterministisch umgesetzt werden, sodass es nur in bestimmten Sonderfällen zu Ungenauigkeiten in der Kollisionsabfrage kommen kann. Und zwar ist das beim \say{\emph{Splitten}} direkt vor Interaktionsobjekten der Fall. Die Kollisionsabfrage des kopierten Spielobjektes des Chronologen ermittelt dabei in seltenen Fällen keine Kollision.

Abgeschlossen wurde diese Arbeit in \say{Kapitel \ref{sec:test}: \nameref{sec:test}} mit Nutzertests, welche das grundlegende Prinzip und die Umsetzung des Prototyps auf das Verstehen der Kernmechanik überprüft haben. Die dabei gewonnenen Aspekte wurden teilweise im Kapitel der Konzeption und der Umsetzung eingebracht. Die restlichen Aspekte werden in der Zukunft der Entwicklung des Spiels einbezogen, um den Prototyp zu verbessern. Durch die Durchführung und Auswertung der Nutzertests konnten ebenso die Forschungsfragen zu einem bestimmten Teil beantwortet werden. Die Komplexität der Rätsel kann in einem ansteigendem und vereinzelt absinkenden Grad weiter entwickelt werden.

\section{Ausblick}\label{sec:ausblick}
Es ist geplant, das Spielkonzept und den Prototyp weiterzuentwickeln. Daher sollten die durch die Nutzertests erkannten Fehler des Programmverhaltens behoben werden, damit das Spielgefühl, das durch die Fehler verschlechtert wurde, erhöht wird.
Zusätzlich müssen weitere Level und neue Szenarien konzipiert und umgesetzt werden, da sie den Hauptteil des Spiels ausmachen. Dazu zählt das Labor, welches der Spieler als \say{Hub} benutzt.

Außerdem soll das Text- und Sprachdialog System überarbeitet und durch Zwischensequenzen ergänzt werden, in denen der Spieler die Haupthandlung verfolgen wird. Zudem fehlt das Konzept, welche Gegenstände und Notizen der Spieler finden und welche Rückschlüsse er daraus ziehen kann. Die dabei entstehenden Hintergrundgeschichten sollen nach einer Weiterentwicklung integriert werden.

Im Zuge dessen kann die \ac{URP} durch die \ac{HDRP} gewechselt werden, um durch das Post-Processing noch mehr grafische Gestaltung zu gewinnen. Um das Entdeckungsgefühl des Spiels und das Aussehen der Levelszenarien auszubauen und feinere grafische Details einzubinden, ist ein Wechsel vom normalen Low-Poly Stil zu einem texturierten Low-Poly Stil eine Möglichkeit. Hierbei können durch Texturen auf den Oberflächen der Spielobjekte, unter anderem an Kanten Details hinzugefügt werden, welche durch den normalen Low-Poly nur aufwendig zu integrieren sind. Ein Beispiel dafür ist das Spiel Ashen (vgl. \cite{gamestar_ashen_2023}), welches einen ähnlichen Art-Stil besitzt.

Um der nicht deterministischen Spiel-Engine vorzubeugen, sollte in Zukunft das \ac{DOTS} System in das Spiel eingebunden werden, um dadurch eine bessere Spielperformanz zu bieten. Infolgedessen können bereits existierende Abläufe der Spielmechanik überarbeitet und leistungsstärker umgesetzt werden.

Das zukünftige Ziel wird es sein, dieses Spiel als ein Indie Spiel zu vermarkten und zu verkaufen.
% Umsetzung von in usertests aufkommenden bugs
% Umsetzung der nur noch konzipierten inhalt
    % zum Beispiel das entdecken sowie das labor
% Ingame Zwischensequezne zussätzlich
% Dialog system
% Integration von DOTS
% weitere Level
% art stil wechsel ähnlich zu ashen um texturielle unterscheidungen zu intgerieren (stylized / textured low poly)
% einhergehen mit wechsel zu hrp um optisch noch mehr im postprocsssing rauszuholen und zu integrieren
% eine weiter entwicklung bis später mal vermarktung 